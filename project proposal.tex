\documentclass{article}
\usepackage{graphicx}
\usepackage{hyperref}
\usepackage{geometry}
\usepackage{booktabs}

\geometry{margin=1in}
\begin{document}

\title{EDS 6340 Introduction to Data Science \\  Project Proposal}
\author{}
\date{\today}
\maketitle

\section*{Project Group Number: 2}
\begin{tabular}{|l|l|}
\hline
\textbf{Name of the Student} & \textbf{UH ID} \\ \hline
Yuzhen Hu & 2299391\\ \hline
Murtaza Mustafa & 2415417 \\ \hline
Duggimpudi, Bala Meghana & 2275900\\ \hline
Rehan, Muhammad Asad & 1951934\\ \hline

\end{tabular}

\section*{Project Title: "Predicting Community Crime Rates Using Socio-Economic and Law Enforcement Data"}

\section*{Dataset Information:}

\begin{tabular}{|l|l|}
\hline
\textbf{Dataset Characteristics} & Multivariate \\ \hline
\textbf{Subject Area} & Social Science\\ \hline
\textbf{Associated Tasks} & Regression \\ \hline
\textbf{Feature Type} & Real \\ \hline
\textbf{Number of Instances} & 2215 \\ \hline
\textbf{Number of Features} & 125 \\ \hline
\textbf{Has Missing Values?} & Yes \\ \hline
\end{tabular}

\noindent \textbf{Dataset:} \url{https://archive.ics.uci.edu/dataset/211/communities+and+crime+unnormalized}

\section*{Quick Summary of the Project:}

The "Communities and Crime Unnormalized" dataset provides a rich collection of socio-economic, demographic, and law enforcement statistics from various U.S. communities, aimed at understanding factors influencing crime rates. In this project, we will develop a regression model to predict crime rates in different communities based on the 125 features provided, which include variables like population demographics, economic status, and law enforcement resources. The project will involve data preprocessing to handle missing values, converting categorical variables into numerical forms, and feature selection using correlation analysis to identify the most influential factors. Multiple regression algorithms, such as linear regression, decision trees, and random forests, will be employed and evaluated to determine the most effective model. The final model will then be tested on unseen data to assess its accuracy and robustness in predicting community crime rates. This study aims to provide insights into which socio-economic and law enforcement characteristics significantly impact crime rates and how they can be leveraged to inform public policy decisions.

\section*{Project Pipeline}
To build a machine learning pipeline for predicting community crime rates based on the provided dataset, we will follow these steps:

\begin{enumerate}
    \item \textbf{Data Collection:} The dataset has been sourced from the UCI Machine Learning Repository. \href{https://archive.ics.uci.edu/dataset/211/communities+and+crime+unnormalized}{Click here} to download the data.

    \item \textbf{Data Preprocessing:} Clean the dataset by handling missing values using appropriate imputation techniques. Convert all categorical features into numerical values using techniques like one-hot encoding. Normalize or standardize the numerical features to ensure they are on a similar scale for optimal model performance.

    \item \textbf{Feature Selection:} Identify and select features that have a significant correlation with crime rates using statistical analysis and correlation plots. This step will help focus on the most influential factors, reducing the risk of overfitting and improving model accuracy.

    \item \textbf{Model Selection:} Choose suitable regression algorithms (e.g., Linear Regression, Decision Trees, Random Forest, Support Vector Regression) to model the relationship between the selected features and community crime rates.

    \item \textbf{Model Training:} Split the dataset into training and testing sets to train the selected regression models. Use cross-validation techniques to ensure that the model's performance is generalized across different subsets of the data.

    \item \textbf{Model Evaluation:} Assess the performance of the models using evaluation metrics such as Mean Absolute Error (MAE), Mean Squared Error (MSE), and R-squared. This will help identify the most effective model for predicting crime rates.

    \item \textbf{Results Interpretation:} Interpret the model's results to understand which socio-economic and law enforcement features have the most significant impact on crime rates. Use the findings to provide insights that can help in policy-making and community resource allocation.
\end{enumerate}

By following these steps, we aim to develop a robust regression model that can effectively predict crime rates in communities based on socio-economic and law enforcement variables.


\section*{Tools, Libraries, and Frameworks:}

\begin{itemize}
    \item NumPy, Pandas for Data Preprocessing
    \item Matplotlib, Seaborn,  for Visualization
    \item Correlation-based Feature Selection
    \item Scikit-learn for Model Building and Evaluation

\end{itemize}

\end{document}
